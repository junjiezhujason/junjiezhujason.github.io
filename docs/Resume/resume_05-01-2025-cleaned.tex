\documentclass[letterpaper,11pt]{article}

\usepackage{latexsym}
\usepackage[empty]{fullpage}
\usepackage{titlesec}
\usepackage{marvosym}
\usepackage[usenames,dvipsnames]{color}
\usepackage{verbatim}
\usepackage{enumitem}
\usepackage[hidelinks]{hyperref}
\usepackage{fancyhdr}
\usepackage[english]{babel}
\usepackage{tabularx}
\input{glyphtounicode}


%----------FONT OPTIONS-------------------
% sans-serif
% \usepackage[sfdefault]{FiraSans}
% \usepackage[sfdefault]{roboto}
% \usepackage[sfdefault]{noto-sans}
% \usepackage[default]{sourcesanspro}

% serif
% \usepackage{CormorantGaramond}
% \usepackage{charter}


\pagestyle{fancy}
\fancyhf{} % clear all header and footer fields
\fancyfoot{}
\renewcommand{\headrulewidth}{0pt}
\renewcommand{\footrulewidth}{0pt}

% Adjust margins
\addtolength{\oddsidemargin}{-0.5in}
\addtolength{\evensidemargin}{-0.5in}
\addtolength{\textwidth}{1in}
\addtolength{\topmargin}{-.5in}
\addtolength{\textheight}{1.0in}

\urlstyle{same}

\raggedbottom
\raggedright
\setlength{\tabcolsep}{0in}

% Sections formatting
\titleformat{\section}{
  \vspace{-4pt}\scshape\raggedright\large
}{}{0em}{}[\color{black}\titlerule \vspace{-5pt}]

% Ensure that generate pdf is machine readable/ATS parsable
\pdfgentounicode=1

%-------------------------
% Custom commands
\newcommand{\resumeItem}[2]{
  \item\small{
    \textbf{#1}{: #2 \vspace{-2pt}}
  }
}

% Just in case someone needs a heading that does not need to be in a list
\newcommand{\resumeHeading}[4]{
    \begin{tabular*}{0.99\textwidth}[t]{l@{\extracolsep{\fill}}r}
      \textbf{#1} & #2 \\
      \textit{\small#3} & \textit{\small #4} \\
    \end{tabular*}\vspace{-5pt}
}

\newcommand{\resumeSubheading}[4]{
  \vspace{-1pt}\item
    \begin{tabular*}{0.97\textwidth}[t]{l@{\extracolsep{\fill}}r}
      \textbf{#1} & #2 \\
      \textit{\small#3} & \textit{\small #4} \\
    \end{tabular*}\vspace{-5pt}
}

\newcommand{\resumeSubSubheading}[2]{
    \begin{tabular*}{0.97\textwidth}{l@{\extracolsep{\fill}}r}
      \textit{\small#1} & \textit{\small #2} \\
    \end{tabular*}\vspace{-5pt}
}

\newcommand{\resumeSubItem}[2]{\resumeItem{#1}{#2}\vspace{-4pt}}

\renewcommand{\labelitemii}{$\circ$}

\newcommand{\resumeSubHeadingListStart}{\begin{itemize}[leftmargin=*]}
\newcommand{\resumeSubHeadingListEnd}{\end{itemize}}
\newcommand{\resumeItemListStart}{\begin{itemize}}
\newcommand{\resumeItemListEnd}{\end{itemize}\vspace{-5pt}}

%-------------------------------------------
%%%%%%  CV STARTS HERE  %%%%%%%%%%%%%%%%%%%%%%%%%%%%


\begin{document}

%----------HEADING-----------------
\begin{tabular*}{\textwidth}{l@{\extracolsep{\fill}}r}
  \textbf{\href{}{\Large Jason (Junjie) Zhu}} & Email: \href{}{junjie.zhu.jason@gmail.com}\\
  \href{}{jasonjunjiezhu.com} & Mobile: \href{}{650-285-7123} \\
\end{tabular*}

%-----------SUMMARY-----------------
\section{Summary}
I am a curiosity-driven and scientifically-trained builder with experience in AI/ML, Statistics, and Graph Algorithms. 
Drawn to hidden patterns, scalable impact, and high-agency teams, I have continuously been applying my skills to real-world problems: 
multi-modal RAGs, search products, biomedical discovery, etc. 

%-----------EDUCATION-----------------
\section{Education}
  \resumeSubHeadingListStart
    \resumeSubheading
      {Stanford University}{Stanford, CA}
      {Ph.D. in Electrical Engineering · M.S. in Statistics}{2014 -- 2020}
    \resumeSubheading
      {Franklin W. Olin College of Engineering}{Needham, MA}
      {B.S. in Electrical and Computer Engineering}{2010 -- 2014}
  \resumeSubHeadingListEnd

%-----------PROFESSIONAL EXPERIENCE-----------------
\section{Experience}
  \resumeSubHeadingListStart

    \resumeSubheading
      {Nexa AI}{Cupertino, CA}
      {Head of AI/ML}{Feb 2025 -- Present}
      \resumeItemListStart
        \resumeItem{Leadership}
          {Leading a lean and fast-paced team to accelerate Gen-AI edge inference on any device.}
        \resumeItem{Local RAGs}
          {Developing privacy-preserving RAGs with small AI models and on-device vision capabilities.}
        \resumeItem{Agentic Systems}
          {Researching action-driven applications with new AI protocols (e.g., MCP, A2A).}
      \resumeItemListEnd

    \resumeSubheading
      {Apple}{Cupertino, CA}
      {Machine Learning Engineer}{Jan 2020 -- Feb 2025}
      \resumeItemListStart
        \resumeItem{Synthetic Data Generation}
          {Invented methods to test model robustness via high-dimensional perturbations.}
        \resumeItem{Preference Learning}
          {Designed cost-efficient offline A/B testing to handle user distribution shifts.}
        \resumeItem{System Evaluation}
          {Implemented production pipelines to evaluate query understanding and ranking systems.}
      \resumeItemListEnd
      
    
      \resumeSubheading
      {Stanford University}{Stanford, CA}
      {Research Assistant}{Sep 2016 -- Feb 2020}
      \resumeItemListStart
        \resumeItem{Graph Visualization}
          {Developed graph visualizations to interpret and analyze the Gene Ontology.}
        \resumeItem{Unsupervised Learning}
          {Created dimension-reduction methods for stem cell and cancer model systems.}
      \resumeItemListEnd
      
    \resumeSubheading
      {Illumina}{San Francisco Bay Area}
      {Deep Learning Scientist (Internship)}{Jun 2017 -- Aug 2017}
      \resumeItemListStart
        \resumeItem{Model Architecture}
          {Combined CNNs, RNNs, and ResNets to improve accuracy for base-calling applications.}
      \resumeItemListEnd

    \resumeSubheading
      {10X Genomics}{Pleasanton, CA}
      {Data Scientist (Internship)}{Jun 2016 -- Aug 2016}
      \resumeItemListStart
        \resumeItem{R/Python Pipelines}
          {Built and productionized pipelines for exploratory single-cell analysis.}
      \resumeItemListEnd

      
      \resumeSubheading
      {Olin College of Engineering}{Needham, MA}
      {Research Assistant}{Sep 2010 -- May 2014}
      \resumeItemListStart
        \resumeItem{Graph Theory}
          {Solved distance-2-based graph coloring problems for special graph families.}
        \resumeItem{Information Theory}
          {Modeled wireless networks with stochastic geometric and interference models.}
      \resumeItemListEnd
      
  \resumeSubHeadingListEnd

%-----------RESEARCH EXPERIENCE-----------------
% \section{Research Experience}
%   \resumeSubHeadingListStart

%     \resumeSubheading
%       {Department of Statistics, Stanford University}{Stanford, CA}
%       {Researcher}{Sep 2016 -- Feb 2020}
%       \resumeItemListStart
%         \resumeItem{Gene Ontology}
%           {Developed graph visualizations to interpret and analyze the Gene Ontology.}
%         \resumeItem{Single-cell RNA-seq}
%           {Implemented exploratory data analysis pipelines for stem-cell and cancer model systems.}
%         \resumeItem{Selective Inference}
%           {Proposed selective inference methods to study tissue-specific expression quantitative trait loci.}
%       \resumeItemListEnd

%     \resumeSubheading
%       {Department of Computer Science, Stanford University}{Stanford, CA}
%       {Researcher}{Sep 2014 -- Sep 2016}
%       \resumeItemListStart
%         \resumeItem{Sequence Alignment}
%           {Designed algorithms in C/C++ to improve speed and accuracy of state-of-the-art sequence aligners for linked-read data.}
%       \resumeItemListEnd

%     \resumeSubheading
%       {Wireless Communication Group, Olin College}{Needham, MA}
%       {Researcher}{Sep 2010 -- May 2014}
%       \resumeItemListStart
%         \resumeItem{SDMA Networks}
%           {Proposed stochastic geometric interference models to extend SDMA network applications.}
%       \resumeItemListEnd

%     \resumeSubheading
%       {Graph Theory Research Group, Olin College}{Needham, MA}
%       {Researcher}{Dec 2011 -- May 2014}
%       \resumeItemListStart
%         \resumeItem{Graph Coloring}
%           {Investigated frequency assignment in wireless networks through graph coloring.}
%         \resumeItem{L(2,1)-labeling}
%           {Found minimum span for L(2,1)-labeling in various graph families.}
%       \resumeItemListEnd

%     \resumeSubheading
%       {Signal Processing and Communication Laboratory, HKUST}{Hong Kong}
%       {Researcher}{Jul 2013 -- Sep 2013}
%       \resumeItemListStart
%         \resumeItem{OFDM Systems}
%           {Studied unsynchronized interferers in multi-antenna OFDM systems.}
%         \resumeItem{Linear Receivers}
%           {Discovered closed-form solutions for typical linear receiver performance.}
%       \resumeItemListEnd

%   \resumeSubHeadingListEnd

%-----------SELECTED PUBLICATIONS-----------------
\section{Selected Publications}
\vspace{0.5em}
  \begin{itemize}[leftmargin=*, topsep=4pt, parsep=2pt, itemsep=1pt]
    \item Automatically Authoring Regression Tests for Machine-Learning-Based Systems. \textit{ICSE, 2021}
    \item Progenitor identification and SARS-CoV-2 infection in human distal lung organoids. \textit{Nature, 2020}
    \item Exploratory gene ontology analysis with interactive visualization. \textit{Scientific Reports, 2019}
    \item Visualization and analysis of sc-RNA-seq data by kernel-based similarity learning. \textit{Nature Methods, 2017}
  \end{itemize}

%-----------GOOGLE SCHOLAR-----------------
See Google Scholar for full list:: \url{}
%-------------------------------------------
\end{document}
