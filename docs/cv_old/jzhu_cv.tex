\documentclass[11pt,letter]{moderncv}

\usepackage{geometry}
 \geometry{
 letterpaper,
 left=1in,
 right=1in,
 top=0.8in,
 bottom=0.8in,
 }% moderncv themes
% \moderncvtheme[blue]{casual}                 % optional argument are 'blue' (default), 'orange', 'red', 'green', 'grey' and 'roman' (for roman %fonts, instead of sans serif fonts)
\moderncvtheme[blue]{classic}                % idem

% character encoding
\usepackage[utf8]{inputenc}     % replace by the encoding you are using
\usepackage{footmisc}
\usepackage{xcolor}
\definecolor{myblue}{RGB}{0, 0, 102}

\renewcommand{\thefootnote}{\fnsymbol{footnote}}
\newcommand{\sstitle}[1]{\textcolor{myblue}{\textbf{\textit{#1}}}}


% adjust the page margins
\setlength{\hintscolumnwidth}{2.5cm}	% if you want to change the width of the column with the dates
%\AtBeginDocument{\setlength{\maketitlenamewidth}{6cm}}  % only for the classic theme, if you want to change the width of your name placeholder (to leave more space for your address details
\AtBeginDocument{\recomputelengths}  % required when changes are made to page layout lengths

% personal data
\firstname{}
\familyname{Junjie (Jason) Zhu}
% \mobile{650-285-7123}
%\email{junjie.zhu.jason@gmail.com}
\email{jjzhu@stanford.edu}
\address{http://jasonjunjiezhu.com}

\nopagenumbers{}  % uncomment to suppress automatic page numbering for CVs longer than one page
%----------------------------------------------------------------------------------
%            content
%----------------------------------------------------------------------------------
\begin{document}

\maketitle

\vspace{-0.1in}
\noindent\rule[0.5ex]{\linewidth}{1pt}
\vspace{0.2in}
%%%%%%%%%%%%%
\section{Education}
\vspace{-0.05in}
\cventry{09/14 - 04/20}{Stanford University}{}{Stanford, CA\newline Ph.D. in Electrical Engineering}{}
%\cventry{09/14 - present}{Ph.D. Candidate in Electrical Engineering}{}{\newline Stanford University, Stanford, CA }{}
{}
\vspace{-0.05in}
\cventry{09/16 - 06/18}{Stanford University}{}{Stanford, CA\newline M.S. in Statistics}{}
%\cventry{09/14 - present}{Ph.D. Candidate in Electrical Engineering}{}{\newline Stanford University, Stanford, CA }{}
{}
\vspace{-0.05in}
\cventry{09/10 - 05/14}{ Franklin W. Olin College of Engineering}{}{Needham, MA \newline B.S. in Electrical and Computer Engineering}{}{}
\vspace{0.05in}
\cvline{}{\sstitle{Relevant Coursework}: Deep Learning for Natural Language Processing; Applied Statistical Learning; Data Visualization; Design and Analysis of Algorithms; Mining Massive Data Sets; Statistical Signal Processing; Convex Optimization.}
\vspace{0.02in}
\cvline{}{\sstitle{Software Development}: Data Mining and Visualization (pandas, D3); Machine Learning (scikit-learn, TensorFlow, Keras); Full-Stack Engineering (Flask, jQuery); Graph-Based Algorithm Design and Data Structures; User Interface Design.}
\vspace{0.02in}
\cvline{}{\sstitle{Programming Languages}: Python, R,  JavaScript, C/C++, bash, MATLAB.}

%%%%%%%%%%%%%
\vspace{0.3in}
\section{Industry Experience}
\vspace{-0.05in}
\cvline{04/20 - present}{\textbf{Machine Learning Engineer},  Apple Inc., Cupertino,  CA
\newline \textit{Manager: Dr. Atif Memon}
\newline Analyze complex data sets with high-dimensional statistics and large-scale computing.
Apply machine learning in information retrieval domain including search engines, natural language processing, or similar areas.
Develop new testing strategies for software services at the intersection of user experience and computational algorithms.
}
\vspace{-0.05in}
\cvline{06/19 - 09/19}{\textbf{Machine Learning Intern},  Apple Inc., Sunnyvale,  CA
\newline \textit{Mentor: Dr. Atif Memon}
\newline Designed a new software testing framework for ML systems, and provided multi-dimensional test coverage for a spelling correction system for search queries.
}
\vspace{-0.05in}
\cvline{06/17 - 09/17}{\textbf{Deep Learning Intern},  Illumina, San Francisco,  CA
\newline \textit{Mentor: Dr. Serafim Batzoglou}
\newline Wrote customized deep learning software in TensorFlow for base-calling applications that systematically combined convolutional neural networks, recurrent neural networks, and residual network architectures to significantly improve accuracy.
% \item Presented single-cell RNA-seq data and analysis results at (i) Festival of Genomics Boston 2016 and (ii) GTEx 2016 Project Community Scientific Meeting
}
\vspace{-0.05in}
\cvline{06/16 - 09/16}{\textbf{Data Science Intern},  10x Genomics, Pleasanton,  CA
\newline \textit{Mentor: Dr. Grace Zheng}
\newline Developed an R software package for exploratory single-cell RNA sequencing analysis and improved internal Python software pipelines.
% \item Presented single-cell RNA-seq data and analysis results at (i) Festival of Genomics Boston 2016 and (ii) GTEx 2016 Project Community Scientific Meeting
}

%%%%%%%%%%%
\vspace{0.01in}
\section{Academic Experience}
\vspace{-0.05in}
\cvline{09/16 - 04/20}{\textbf{Research Assistant},  School of Medicine,  Stanford University
\newline \textit{Advisor: Prof. Chiara Sabatti}
 {\newline
Developed graph visualizations to interpret and analyze the Gene Ontology. Implemented single-cell RNA sequencing exploratory data analysis pipelines for stem-cell and cancer model systems. Proposed selective inference methods to study tissue-specific expression quantitative trait loci.
}}
\vspace{-0.05in}
\cvline{09/14 - 09/16}{\textbf{Research Assistant},  Department of Computer Science,  Stanford University
\newline
\textit{Advisor: Prof. Serafim Batzoglou}
 {
 \newline
Designed algorithms in C and C++ to improve both the speed and accuracy of current state-of-the-art sequence aligners when aligning linked-read data.
}}


\vspace{-0.05in}
\cvline{09/10 - 05/14}{\textbf{Research Assistant}, Wireless Communication Group, Olin College
\newline \textit{Advisor: Prof. Siddhartan Govindasamy}
{
\newline
Proposed  stochastic geometric interference models to extend the range of applications in Space-Division Multiple Access (SDMA) networks.
}}
\vspace{-0.05in}
\cvline{12/11 - 05/14}{\textbf{Research Assistant}, Graph Theory Research Group,  Olin College
\newline \textit{Advisors: Prof. Sarah S. Adams and Prof. Denise S. Troxell}
\newline
Investigated scenarios of frequency assignment in wireless networks through graph coloring and found the minimum span of the $L(2,1)$-labeling for representative families of graphs, \textit{e.g.}, amalgamation graphs and edge-path replacement graphs.
}
\vspace{-0.05in}
\cvline{07/13 - 09/13}{\textbf{Research Intern}, Signal Processing and Communication Laboratory, Hong Kong University of Science and Technology
\newline \textit{Advisor: Prof. Matthew R. McKay}
\newline
Studied the impact of unsynchronized (in frequency) interferers in multi-antenna Orthogonal Frequency-Division Multiplexing (OFDM) systems and discovered closed-form solutions for the performance of typical linear receivers.
}

%%%%%%%%%%%%%
\vspace{0.05in}
\section{Honors and Awards}
\cventry{05/18}{Invited Talk}{invited to present work on cancer immuno-profiling analysis at the annual Stanford Center of Cancer Systems Biology Symposium}{}{\textit{Stanford, CA}}{}
\cventry{04/18}{Popular Poster Award}{voted as the most popular poster at the annual Stanford Biomedical Computation Symposium}{}{\textit{Stanford, CA}}{}
\cventry{01/18}{ NSF Student Travel Grant}{}{awarded for the IPAM workshop on ``Algorithmic Challenges in Protecting Privacy for Biomedical Data''}{\textit{Los Angeles, CA}}{}
\cventry{09/14}{Stanford Graduate Fellowship}{}{three-year graduate fellowship for doctoral studies at Stanford University}{\textit{Stanford, CA}}{}
\cventry{06/14}{IEEE Communications Society Student Travel Grant}{}{awarded for travel to the IEEE International Conference on Communications 2014}{\textit{Sydney, Australia}}{}
\cventry{01/13}{Mathematical Association of America Student Travel Grant}{}{awarded for travel to the Joint Mathematics Meetings 2014}{\textit{Baltimore, MD}}{}
\cventry{06/13}{NSF Student Travel Grant}{}{awarded for travel to the IEEE International Conference on Communications 2013}{\textit{Budapest, Hungary}}{}
\cventry{06/12}{IEEE Communications Society Student Travel Grant}{}{awarded for travel to the IEEE International Conference on Communications 2012}{\textit{Ottawa, Canada}}{}
\cventry{09/10}{Franklin W. Olin  Merit Scholarship}{}{four-year half-tuition scholarship awarded for undergraduate studies at Olin College}{\textit{Needham, MA}}{}


%%%%%%%%%%%%%
\vspace{0.1in}
\section{Professional Service and Leadership}
\cvline{11/14 - 11/16}{ \textbf{Vice Chair} of Cross-disciplinary Healthcare Innovation Partnerships at Stanford
\newline
{\small Coordinated networking events with healthcare startups and Stanford graduate students, and managed social media and website for the student organization. }
 }

\cvline{06/11 - 06/13}{\textbf{Chair} of IEEE Olin Student Branch
\newline
{\small Organized board meetings, adjudicated responsibilities and led fund-raising. Hosted monthly events for IEEE Power Energy Society and IEEE Consultants' Network. }
 }
\cvline{06/13 - present}{\textbf{Reviewer}  for
\begin{itemize}
\item
2019 International Society for Computational Biology (ISMB)
\item
2018 International Society for Computational Biology (ISMB)
\item
2016 Annual Conference on Neural Information Processing Systems (NIPS)
\item
2015 International Conference on Bioinformatics (InCoB)
\item
2014 IEEE Vehicular Technology Conference  (VTC)
\item
IEEE Communications Letters
\item
Journal of Combinatorial Optimization
\item
Discrete Applied Mathematics
\end{itemize}}
\vspace{-0.2in}

% \newpage
\vspace{0.05in}

%%%%%%%%%%%%%
\vspace{0.3in}
\section{Teaching}

\cvline{}{{\large \textbf{{Graduate Teaching Assistant}}}, Stanford University,\textit{ Stanford, CA}}
\vspace{0.02in}
\cvline{Winter 2016}{\textit{CS262 Computational Genomics}
\newline
{\small Prepared and lectured topics in both Hidden Markov Models and Single-Cell Technologies. Wrote problem sets and solutions, and held office hours to review class material. Coordinated two other teaching assistants to grade problem sets.}}
\cvline{}{{\large \textbf{{Undergraduate Head Teaching Assistant}}}, Olin College,\textit{ Needham, MA}}
\vspace{0.02in}
\cvline{Spring 2013}{\textit{ENGR2410 Signals and Systems}
\newline
{\small Prepared lecture notes,  problem sets, quizzes and solutions with the instructor. Held tutorials to review class material. Coordinated three teaching assistants to grade problem sets and quizzes. Helped students with final class projects.}}

\cvline{Spring 2012}{\textit{MATH2188 Special Topics in Mathematics: Linearity}
\newline
{\small Assisted instructors in curriculum design,  integrating Linear Algebra and Differential Equations. Reviewed text book drafts and created answer keys. Coordinated six teaching  assistants to grade quizzes and hold tutorials for a class of over 80 students.
}}
\vspace{0.1in}
\cvline{}{{\large \textbf{{Undergraduate Teaching Assistant}}}, Olin College,\textit{ Needham, MA}}
\vspace{0.02in}

\cvline{Fall 2013}{\textit{ENGR3420 Introduction to Analog and Digital Communications}
\newline
{\small Graded problem sets and created solution keys. Held office hours and review sessions. Gave one lecture on an advanced topic on wireless networks. }}

\cvline{Spring 2013}{\textit{MATH2199 An Introduction to Random Graphs}
\newline
{\small Reviewed and distributed reading materials. Helped students with independent study through individual meetings. Gave feedback to the instructor. }}

\cvline{Fall 2012}{\textit{MATH2110 Discrete Mathematics}
\newline
{\small  Assisted visiting professor during class time and gave feedback on lectures. Graded problems sets and held review sessions. Presented research in graph theory.}}

\cvline{Fall 2011}{\textit{MATH1111 Modeling and Simulation of the Physical World}
\newline
{\small Taught first-year students basics of programming in MATLAB. Met with individual students to check off weekly assignments. Gave project feedback to students.
}}

%%%%%%%%%%%%%
\vspace{0.5in}
\section{Publications}
\vspace{-0.05in}
\cvline{Computational Biology}{
\begin{enumerate}
\item
``Exploratory gene ontology analysis with interactive visualization'',  \textbf{J. Zhu}, Q. Zhao, E. Katsevich, C. Sabatti, \textit{Scientific Reports}, Vol 1, No. 1, pp.7793, 2019.
\item
``Selection-adjusted inference: an application to confidence intervals for cis-eQTL effect sizes''. S. Panigrahi, \textbf{J. Zhu}, C. Sabatti, \textit{accepted to Biostatistics}.
\item
``Organoid modeling of the tumor immune microenvironment''. J. T. Neal$^\dagger$, X. Li$^\dagger$, \textbf{J. Zhu}, ...,  C. Sabatti, J. S. Boehm, W. C. Hahn, G. X.Y. Zheng, M. M. Davis, C. J. Kuo, \textit{Cell}, Vol. 175, No. 7, pp. 1972-1988, 2018.
\item
``Inflammatory cytokine TNF$\alpha$ promotes the long-term expansion of primary hepatocytes in 3D culture''. W. C. Peng, C. Y. Logan, M. Fish, T. Anbarchian, F. Aguisanda, A. Alvarez-Varela, P. Wu, Y. Jin, \textbf{J Zhu}, B. Li, M. Grompe, B. Wang, R. Nusse, \textit{Cell}, Vol. 175, No. 6, pp. 1607-1619, 2018.
\item
``Network enhancement: a general method to denoise weighted biological networks''. B. Wang$^\dagger$, A. Pourshafeie$^\dagger$, M. Zitnik$^\dagger$, \textbf{J Zhu}, C. D. Bustamante, J. Leskovec, S. Batzoglou, \textit{Nature Communications}, Vol. 9, No. 1, pp. 3108, 2018.
\item
``SIMLR: a tool for large-scale genomic analyses by multi-kernel learning''. B. Wang, D. Ramazzotti, L. De Sano, \textbf{J. Zhu}, E. Pierson, S. Batzoglou.\textit{Proteomics}, Vol. 18,  No. 2, pp. 1700232, 2018.
\item
``scRNASeqDB: a database for gene expression profiling in human single cell by RNA-seq'', Y. Cao$^\dagger$, \textbf{J. Zhu}$^\dagger$, G. Han, P. Jia, Z. Zhao, \textit{Genes}, Vol. 8, No. 12, pp. 368, 2017.
\item
``Visualization and analysis of single-cell RNA-seq data by kernel-based similarity learning'',  B. Wang, \textbf{J. Zhu}, E. Pierson, D. Ramazzotti, S. Batzoglou,  \textit{Nature Methods}, Vol. 14, No. 4, pp. 414, 2017.
\item
``Massively parallel digital transcriptional profiling of single cells''  G. X.Y. Zheng, J. M Terry, P. Belgrader, P. Ryvkin, Z. W. Bent, R. Wilson, S. B. Ziraldo, T. D. Wheeler, G. P. McDermott, \textbf{J. Zhu}, ...,  T. S. Mikkelsen, B. J. Hindson,  J. H. Bielas,  \textit{Nature Communications}, Vol. 10, No. 11, pp. 1096-1098, 2017.
\item
``Unsupervised learning from noisy networks with applications to Hi-C data''  B. Wang, \textbf{J. Zhu}, Oana Ursu, Armin Pourshafeie, Serafim Batzoglou, Anshul Kundaje, in Proceedings of  \textit{Advances in Neural Information Processing Systems (NIPS)}, Barcelona, 2016.
\end{enumerate}
}

\vspace{-0.1in}
\cvline{Signal Processing}{
\begin{enumerate}
\item
``Performance of multiantenna linear MMSE receivers in doubly stochastic networks.'' \textbf{J. Zhu}, S. Govindasamy, J. Hwang,  \textit{IEEE Transactions on Communications},  vol. 62, no. 8, pp. 2825-2839, Aug. 2014.
\item
``On the impact of unsynchronized interferers on multi-antenna OFDM systems.''  \textbf{J. Zhu}, R. H. Y. Louie, M. R. McKay, S. Govindasamy,   in Proceedings of  \textit{the IEEE International Conference on Communications (ICC)}, Sydney, 2014.
\item
``Performance of multi-antenna linear MMSE receivers in the presence of clustered interferers.''  \textbf{J. Zhu}, S. Govindasamy,  in Proceedings of \textit{the IEEE International Conference on Communications (ICC)}, Budapest, 2013.
\item
``Performance of multi-antenna MMSE receivers in non-homogenous Poisson networks.''  \textbf{J. Zhu}, S. Govindasamy,  in Proceedings of \textit{the IEEE International Conference on Communications (ICC)}, Ottawa, 2012.
\end{enumerate}
}
\footnotetext{\vspace{-0.1in} \newline $\dagger$ These authors contributed equally.}

% \vspace{1.9in}
\cvline{Graph Theory}{
\begin{enumerate}
\item
``The minimum span of L(2,1)-labelings of generalized flowers.'' N. Karst, J. Oehrlein, D. S. Troxell,  \textbf{J. Zhu}*, \textit{Discrete Applied Mathematics}, Vol. 181, pp. 139-151, January 2015.
\item
``Labeling amalgamations of Cartesian products of complete graphs with a condition at distance two.'' N. Karst, J. Oehrlein, D. S. Troxell, \textbf{J. Zhu}*, \textit{Discrete Applied Mathematics}, Vol. 178, pp. 101-108, December, 2014.
\item
``On distance labelings of amalgamations and injective labelings of general graphs." N. Karst, J. Oehrlein, D. S. Troxell,  \textbf{J. Zhu}*, \textit{Involve, a Journal of Mathematics}, Vol. 8, No. 4, pp. 535-540, 2014.
\item
``$L(d,1)$-labelings of the edge-path-replacement by factorization of graphs.''  N. Karst, J. Oehrlein, D. S. Troxell,  \textbf{J. Zhu}*, \textit{Journal of Combinatorial Optimization}, 2013.
\item
``On the $L(2, 1)$-labelings of amalgamations of graphs.'' S. S. Adams, N. Howell, N. Karst, D. S. Troxell, \textbf{J. Zhu}*, \textit{Discrete Applied Mathematics}, Vol. 161, No. 7-8, pp. 881-888, May 2013.
\end{enumerate}
}
\footnotetext{\vspace{-0.1in}  \newline
* Authors for mathematics journals are listed in alphabetical order.}


%%%%%%%%%%%%%
%\vspace{0.1in}
%\section{Selected Course Projects}
%
%\cventry{Winter 2017}{Machine Comprehension}{Natural Language Processing with Deep Learning}{Stanford University}{}{Implemented a model using bidirectional long short-term memory (BiLSTM) encoding of question and context followed by a co-attention model for the SQuAD dataset. Extended the model by adding Dropout and randomization strategies to account for unknown tokens.}
%
%\cventry{Spring 2016}{Mining GTEx Data}{Mining Massive Data Sets}{Stanford University}{}{Led a team of three students to apply different types of LASSO regression to predict Gene Ontology (GO) functions using tissue-specific gene expression data collected by the GTEx consortium. Studied complementary and substitutional effects in predicting these tissue-specific functions. }
%
%\cventry{Spring 2014}{Spectral Imaging for CT}{Senior Capstone Project}{Olin College}{}{Worked on a year-long company project sponsored by Analogic Corporation on a five-person team. Developed digital signal processing algorithms which optimize the quality of the image output by an X-ray computed tomography (CT) system. }
%
%\cventry{Fall 2012}{OFDM with USRP}{Wireless Communications}{Olin College}{}{Implemented a one-to-one link under QPSK and 8-PSK modulation schemes with Orthogonal Frequency-Division Multiplexing (OFDM) using the Universal Software Radio Peripheral (USRP), and accurately transferred data after timing synchronization and channel equalization.}
%
%\cventry{Fall 2012}{Digital Echo Canceller}{Digital Signal Processing}{Olin College}{}{Investigated fundamentals of adaptive filters and implemented acoustic echo cancellers in MATLAB and LabVIEW using the time-domain and the frequency-domain least-mean-square algorithms.}
%
%\cventry{Spring 2012}{Wireless Power Transfer}{Signals and Systems}{Olin College}{}{Self-studied principles of power transfer in magnetic coupled circuits, compared two different models: capacitors in series and parallel with inductors by circuit simulation, and found the optimal load resistance.}
%
%
%\cventry{Spring 2011}{Brainwave-controlled Game}{Modeling and Controls}{Olin College}{}{Designed a two-player brainwave battle game:  built filter and amplification circuits, modified electro-node sensors for clean real-time measurement of low-beta brainwaves, manipulated fan speed with analog output through Arduino controls.}
%
%\cventry{Spring 2011}{Stock Market Prediction Program}{Software Design}{Olin College}{}{Created a Python program with the ability to draw correlation between stocks with link analysis, developed prediction models validated by historical stock market data and designed an interactive user-friendly GUI.}







\end{document}

