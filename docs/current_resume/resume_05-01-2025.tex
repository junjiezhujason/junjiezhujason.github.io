\documentclass[letterpaper,11pt]{article}

\usepackage{latexsym}
\usepackage[empty]{fullpage}
\usepackage{titlesec}
\usepackage{marvosym}
\usepackage[usenames,dvipsnames]{color}
\usepackage{verbatim}
\usepackage{enumitem}
\usepackage[hidelinks]{hyperref}
\usepackage{fancyhdr}
\usepackage[english]{babel}
\usepackage{tabularx}
\usepackage{ragged2e}
\input{glyphtounicode}


%----------FONT OPTIONS-------------------
% sans-serif
% \usepackage[sfdefault]{FiraSans}
% \usepackage[sfdefault]{roboto}
% \usepackage[sfdefault]{noto-sans}
% \usepackage[default]{sourcesanspro}

% serif
% \usepackage{CormorantGaramond}
% \usepackage{charter}


\pagestyle{fancy}
\fancyhf{} % clear all header and footer fields
\fancyfoot{}
\renewcommand{\headrulewidth}{0pt}
\renewcommand{\footrulewidth}{0pt}

% Adjust margins
\addtolength{\oddsidemargin}{-0.5in}
\addtolength{\evensidemargin}{-0.5in}
\addtolength{\textwidth}{1in}
\addtolength{\topmargin}{-.5in}
\addtolength{\textheight}{1.0in}

\urlstyle{same}

\raggedbottom
\raggedright
\setlength{\tabcolsep}{0in}

% Sections formatting
\titleformat{\section}{
  \vspace{-4pt}\scshape\raggedright\large
}{}{0em}{}[\color{black}\titlerule \vspace{-5pt}]



% Ensure that generate pdf is machine readable/ATS parsable
\pdfgentounicode=1

%-------------------------
% Custom commands
\newcommand{\resumeItem}[2]{
  \item\small{
    \textbf{#1}{: #2 \vspace{-2pt}}
  }
}

% Just in case someone needs a heading that does not need to be in a list
\newcommand{\resumeHeading}[4]{
    \begin{tabular*}{0.99\textwidth}[t]{l@{\extracolsep{\fill}}r}
      \textbf{#1} & #2 \\
      \textit{\small#3} & \textit{\small #4} \\
    \end{tabular*}\vspace{-5pt}
}

\newcommand{\resumeSubheading}[4]{
  \vspace{-1pt}\item
    \begin{tabular*}{0.97\textwidth}[t]{l@{\extracolsep{\fill}}r}
      \textbf{#1} & #2 \\
      \textit{\small#3} & \textit{\small #4} \\
    \end{tabular*}\vspace{-5pt}
}

\newcommand{\resumeSubSubheading}[2]{
    \begin{tabular*}{0.97\textwidth}{l@{\extracolsep{\fill}}r}
      \textit{\small#1} & \textit{\small #2} \\
    \end{tabular*}\vspace{-5pt}
}

\newcommand{\resumeSubItem}[2]{\resumeItem{#1}{#2}\vspace{-4pt}}

\renewcommand{\labelitemii}{$\circ$}

\newcommand{\resumeSubHeadingListStart}{\begin{itemize}[leftmargin=*]}
\newcommand{\resumeSubHeadingListEnd}{\end{itemize}}
\newcommand{\resumeItemListStart}{\begin{itemize}}
\newcommand{\resumeItemListEnd}{\end{itemize}\vspace{-5pt}}

%-------------------------------------------
%%%%%%  CV STARTS HERE  %%%%%%%%%%%%%%%%%%%%%%%%%%%%

% Add this after the color package
\definecolor{keywordcolor}{RGB}{0, 90, 160}

\begin{document}

%----------HEADING-----------------
\begin{tabular*}{\textwidth}{l@{\extracolsep{\fill}}r}
  \textbf{\href{}{\Large Jason (Junjie) Zhu, Ph.D.}} & Email: \href{}{junjie.zhu.jason@gmail.com}\\
  \href{https://jasonjunjiezhu.com}{https://jasonjunjiezhu.com} & Mobile: \href{tel:+16502857123}{650-285-7123} \\
\end{tabular*}

%-----------SUMMARY-----------------

\section{Summary}
\begin{justify}
  I am a curiosity driven, scientifically trained builder with over 10 years of experience in AI/ML, statistics, and graph algorithms. My academic work has received over 10,000 citations, and beyond research, I’ve led and built production systems from the ground up: designing systems, writing production code, shaping direction, and delivering impact alongside top tier product teams. These include scalable systems in multi modal RAGs, intelligent search, and biomedical discovery.
  I value clear, thoughtful communication as much as technical precision, and care deeply about getting the details right. 
  I’ve found my greatest motivation comes not just from building things that work, but from helping people grow, especially those who bring diverse perspectives and backgrounds. 
\end{justify}

%-----------EDUCATION-----------------
\section{Education}
\begin{tabular*}{\textwidth}{@{\extracolsep{\fill}}p{0.65\textwidth}r}
\textbf{Stanford University} & Stanford, CA \\
\textit{\small Ph.D. in Electrical Engineering · M.S. in Statistics} & \textit{\small 2014 -- 2020} \\[0.4em]
\textbf{Olin College of Engineering} & Needham, MA \\
\textit{\small B.S. in Electrical and Computer Engineering} & \textit{\small 2010 -- 2014} \\
\end{tabular*}


\section{Experience}
\begin{tabular*}{0.97\textwidth}[t]{l@{\extracolsep{\fill}}r}
  \textbf{Nexa AI} & Cupertino, CA \\
  \textit{\small Head of AI/ML} & \textit{\small Feb 2025 -- Present} \\
  \multicolumn{2}{l}{
    \begin{minipage}{\textwidth}
      \vspace{0.3em}
      \begin{itemize}[leftmargin=*, itemsep=-4.5pt, topsep=0pt, label={\raisebox{0.4ex}{\tiny\textbullet}}]
        \item {\bf RAG Productization}: Led a team of four to develop and launch an on-device RAG application in under three months, leveraging continuous regression testing and weekly iterations to accelerate quality improvements.
        \item {\bf Semantic Search Innovation}: Invented, prototyped, and deployed a semantic file-search system with structured metadata, vector search, and @-search support, significantly improving file recall quality.
      \end{itemize}
    \end{minipage}
  } \\
  \noalign{\vspace{0.7em}}

  \textbf{Apple} & Cupertino, CA \\
  \textit{\small Machine Learning Engineer} & \textit{\small Feb 2020 -- Feb 2025} \\
  \multicolumn{2}{l}{
    \begin{minipage}{\textwidth}
      \vspace{0.3em}
      \begin{itemize}[leftmargin=*, itemsep=-4.5pt, topsep=0pt, label={\raisebox{0.4ex}{\tiny\textbullet}}]       
        \item {\bf Offline Evaluation}: Spearheaded the development and rollout of an offline evaluation service to rigorously quantify feature impact on user experience prior to public releases, enabling faster improvement cycles. Owned multi-year roadmap strategy and stakeholder alignment. Scaled the initiative from a solo effort to a five-person team, empowering members to independently drive new scopes and sustain the roadmap beyond my tenure.
        \item {\bf Research Innovation}: Designed novel generative and retrieval-based frameworks to evaluate million-scale daily traffic in Apple Maps Search; presented methodologies at top-tier software engineering conferences ({\it ICSE}, {\it FSE}).
        \item {\bf Infrastructure Modernization}: Revamped internal testing pipelines for query understanding and ranking, reducing release cycles from weekly to daily and improving launch stability for WWDC-featured products.
        \item {\bf Technical Leadership}: Mentored team members in defining project scopes and preparing presentations for internal AI/ML conferences, fostering professional growth and cross-team visibility.
      \end{itemize}
    \end{minipage}
  } \\
  \noalign{\vspace{0.7em}}

  \textbf{Stanford University} & Stanford, CA \\
  \textit{\small Research Assistant} & \textit{\small Sep 2014 -- Feb 2020} \\
  \multicolumn{2}{l}{
    \begin{minipage}{\textwidth}
      \vspace{0.3em}
      \begin{itemize}[leftmargin=*, itemsep=-4.5pt, topsep=0pt, label={\raisebox{0.4ex}{\tiny\textbullet}}]
        \item \textbf{Full-Stack Data Science}: Developed an interactive tool to visualize and perform power analysis on 30,000+ Gene Ontology terms, enabling large-scale association discovery with controlled false discovery rate.
        \item {\bf Scalable Graph Learning}: Built graph-based unsupervised learning pipelines for million-scale, high-dimensional datasets, resulting in publications in {\it Nature}, {\it Nature Methods}, {\it Cell}, and {\it NeurIPS}.
      \end{itemize}
    \end{minipage}
  } \\
  \noalign{\vspace{0.7em}}

  % \textbf{Olin College of Engineering} & Needham, MA \\
  % \textit{\small Undergraduate Researcher} & \textit{\small Sep 2010 -- May 2014} \\
  % \multicolumn{2}{l}{
  %   \begin{minipage}{\textwidth}
  %     \vspace{0.3em}
  %     % TODO: combine these into one bullet point
  %     \begin{itemize}[leftmargin=*, itemsep=-4.5pt, topsep=0pt, label={\raisebox{0.4ex}{\tiny\textbullet}}]
  %       \item {\bf Theoretical Research}: Developed combinatorial algorithms for graph coloring and stochastic geometric models for wireless interference, leading to 8 publications—5 in discrete math journals and 3 in flagship IEEE conferences.
  %     \end{itemize}
  %   \end{minipage}
  % } \\
\end{tabular*}

%-----------SELECTED PUBLICATIONS-----------------
\section{Selected Publications}
\vspace{0.5em}
  \begin{enumerate}[leftmargin=*, topsep=4pt, parsep=2pt, itemsep=1pt]
    \item Automatically Authoring Regression Tests for Machine-Learning-Based Systems. \textit{ICSE, 2021}
    \item Progenitor identification and SARS-CoV-2 infection in human distal lung organoids. \textit{Nature, 2020}
    % \item Exploratory gene ontology analysis with interactive visualization. \textit{Scientific Reports, 2019}
    \item Visualization and analysis of sc-RNA-seq data by kernel-based similarity learning. \textit{Nature Methods, 2017}
  \end{enumerate}
Full list shown on Google Scholar: \url{https://scholar.google.com/citations?user=2EasRdEAAAAJ&hl}
%-------------------------------------------
\end{document}
