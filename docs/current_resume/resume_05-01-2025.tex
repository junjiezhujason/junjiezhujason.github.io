\documentclass[letterpaper,11pt]{article}

\usepackage{latexsym}
\usepackage[empty]{fullpage}
\usepackage{titlesec}
\usepackage{marvosym}
\usepackage[usenames,dvipsnames]{color}
\usepackage{verbatim}
\usepackage{enumitem}
\usepackage[hidelinks]{hyperref}
\usepackage{fancyhdr}
\usepackage[english]{babel}
\usepackage{tabularx}
\usepackage{ragged2e}
\input{glyphtounicode}


%----------FONT OPTIONS-------------------
% sans-serif
% \usepackage[sfdefault]{FiraSans}
% \usepackage[sfdefault]{roboto}
% \usepackage[sfdefault]{noto-sans}
% \usepackage[default]{sourcesanspro}

% serif
% \usepackage{CormorantGaramond}
% \usepackage{charter}


\pagestyle{fancy}
\fancyhf{} % clear all header and footer fields
\fancyfoot{}
\renewcommand{\headrulewidth}{0pt}
\renewcommand{\footrulewidth}{0pt}

% Adjust margins
\addtolength{\oddsidemargin}{-0.5in}
\addtolength{\evensidemargin}{-0.5in}
\addtolength{\textwidth}{1in}
\addtolength{\topmargin}{-.5in}
\addtolength{\textheight}{1.0in}

\urlstyle{same}

\raggedbottom
\raggedright
\setlength{\tabcolsep}{0in}

% Sections formatting
\titleformat{\section}{
  \vspace{-4pt}\scshape\raggedright\large
}{}{0em}{}[\color{black}\titlerule \vspace{-5pt}]

% Ensure that generate pdf is machine readable/ATS parsable
\pdfgentounicode=1

%-------------------------
% Custom commands
\newcommand{\resumeItem}[2]{
  \item\small{
    \textbf{#1}{: #2 \vspace{-2pt}}
  }
}

% Just in case someone needs a heading that does not need to be in a list
\newcommand{\resumeHeading}[4]{
    \begin{tabular*}{0.99\textwidth}[t]{l@{\extracolsep{\fill}}r}
      \textbf{#1} & #2 \\
      \textit{\small#3} & \textit{\small #4} \\
    \end{tabular*}\vspace{-5pt}
}

\newcommand{\resumeSubheading}[4]{
  \vspace{-1pt}\item
    \begin{tabular*}{0.97\textwidth}[t]{l@{\extracolsep{\fill}}r}
      \textbf{#1} & #2 \\
      \textit{\small#3} & \textit{\small #4} \\
    \end{tabular*}\vspace{-5pt}
}

\newcommand{\resumeSubSubheading}[2]{
    \begin{tabular*}{0.97\textwidth}{l@{\extracolsep{\fill}}r}
      \textit{\small#1} & \textit{\small #2} \\
    \end{tabular*}\vspace{-5pt}
}

\newcommand{\resumeSubItem}[2]{\resumeItem{#1}{#2}\vspace{-4pt}}

\renewcommand{\labelitemii}{$\circ$}

\newcommand{\resumeSubHeadingListStart}{\begin{itemize}[leftmargin=*]}
\newcommand{\resumeSubHeadingListEnd}{\end{itemize}}
\newcommand{\resumeItemListStart}{\begin{itemize}}
\newcommand{\resumeItemListEnd}{\end{itemize}\vspace{-5pt}}

%-------------------------------------------
%%%%%%  CV STARTS HERE  %%%%%%%%%%%%%%%%%%%%%%%%%%%%


\begin{document}

%----------HEADING-----------------
\begin{tabular*}{\textwidth}{l@{\extracolsep{\fill}}r}
  \textbf{\href{}{\Large Jason (Junjie) Zhu, Ph.D.}} & Email: \href{}{junjie.zhu.jason@gmail.com}\\
  \href{}{jasonjunjiezhu.com} & Mobile: \href{}{650-285-7123} \\
\end{tabular*}

%-----------SUMMARY-----------------

\section{Summary}
\begin{justify}
I am a curiosity-driven, scientifically trained builder with 10+ years of experience in AI/ML, statistics, and graph algorithms. 
I have had the fortune to collaborate with world-class researchers and top-tier product teams to drive meaningful, collective impact—reflected in 10,000+ citations to my publications. 
Passionate about complex challenges and high-agency environments, I architect and implement scalable solutions across emerging domains, from multi-modal RAGs and intelligent search to biomedical discovery.
\end{justify}

%-----------EDUCATION-----------------
\section{Education}
\begin{tabular*}{\textwidth}{@{\extracolsep{\fill}}p{0.65\textwidth}r}
\textbf{Stanford University} & Stanford, CA \\
\textit{\small Ph.D. in Electrical Engineering · M.S. in Statistics} & \textit{\small 2014 -- 2020} \\[0.4em]
\textbf{Olin College of Engineering} & Needham, MA \\
\textit{\small B.S. in Electrical and Computer Engineering} & \textit{\small 2010 -- 2014} \\
\end{tabular*}
%-----------PROFESSIONAL EXPERIENCE-----------------
\section{Experience}
\begin{tabular*}{0.97\textwidth}[t]{l@{\extracolsep{\fill}}r}
  \textbf{Nexa AI} & Cupertino, CA \\
  \textit{\small Head of AI/ML} & \textit{\small Feb 2025 -- Present} \\
  \multicolumn{2}{l}{
    \begin{minipage}{\textwidth}
      \vspace{0.3em}
      \begin{itemize}[leftmargin=*, itemsep=-4.5pt, topsep=0pt, label={\raisebox{0.4ex}{\tiny\textbullet}}]
        \item Semantic Search Innovation: Invented a local file-search semantic search engine with structured metadata support and @-search features, enhancing real-time query resolution and usability for customer-facing demos.
        \item From 0 to 1: Led a 4-member team to build and ship an on-device RAG system in under 3 months, powered by continuous regression testing from day one to support rapid iteration and measurable weekly quality gains.
      \end{itemize}
    \end{minipage}
  } \\
  \noalign{\vspace{0.7em}}

  \textbf{Apple} & Cupertino, CA \\
  \textit{\small Machine Learning Engineer} & \textit{\small Jan 2020 -- Feb 2025} \\
  \multicolumn{2}{l}{
    \begin{minipage}{\textwidth}
      \vspace{0.3em}
      \begin{itemize}[leftmargin=*, itemsep=-4.5pt, topsep=0pt, label={\raisebox{0.4ex}{\tiny\textbullet}}]       
        \item Modernized internal evaluation pipelines for query understanding and ranking services to accelerate software deployment velocity (weekly to daily), enhancing the quality and stability of new features showcased at WWDC.
        \item Developed both generative and retrieval-based methods to evaluate ML systems at scale, applied them to industrial-scale search engines and shared the early results at top software engineering conferences ({\it ICSE}, {\it FSE}).
        \item Built project roadmaps and reduced manual triaging time by nearly 50\% year over year, building capacity for collegues to innovate on new evaluation inniatives and achieve recognition through internal AI/ML conferences.
      \end{itemize}
    \end{minipage}
  } \\
  \noalign{\vspace{0.7em}}

  \textbf{Stanford University} & Stanford, CA \\
  \textit{\small Graudate Research Assistant} & \textit{\small Sep 2014 -- Feb 2020} \\
  \multicolumn{2}{l}{
    \begin{minipage}{\textwidth}
      \vspace{0.3em}
      \begin{itemize}[leftmargin=*, itemsep=-4.5pt, topsep=0pt, label={\raisebox{0.4ex}{\tiny\textbullet}}]
        \item Full-Stack: developed an interactive tool to visualize 30,000+ Gene Ontology terms and 70,000+ genes, enabling power analyses and simulations across high-throughput genomic data to control the false discovery rate.
        % \item Created  a graph-based feature selection framework integrating spatial data, achieving over $>$95 true positive alignment with experimental resutls while controlling false discovery rates under 5\%.
        \item Applied Research: Built upon graph-based unsupervised learning methods to develop a scalable pipelines with million-scale sample-size data sets, resulting in publications in top scientific journals ({\it Nature}, {\it Nature Methods}, {\it Cell}, {\it NeurIPS}).
      \end{itemize}
    \end{minipage}
  } \\
  \noalign{\vspace{0.7em}}

  \textbf{Olin College of Engineering} & Needham, MA \\
  \textit{\small Undergraduate Researcher} & \textit{\small Sep 2010 -- May 2014} \\
  \multicolumn{2}{l}{
    \begin{minipage}{\textwidth}
      \vspace{0.3em}
      % TODO: combine these into one bullet point
      \begin{itemize}[leftmargin=*, itemsep=-4.5pt, topsep=0pt, label={\raisebox{0.4ex}{\tiny\textbullet}}]
        \item Theoretical Research: Developed combinatorial algorithms for distance-2 graph coloring and derived analytical models for wireless network interference using stochastic geometry, resulting in 8 peer-reviewed publications—including 3 first-author papers in flagship IEEE venues and 5 in discrete math journals.      \end{itemize}
    \end{minipage}
  } \\
\end{tabular*}

%-----------SELECTED PUBLICATIONS-----------------
\section{Selected Publications}
\vspace{0.5em}
  \begin{enumerate}[leftmargin=*, topsep=4pt, parsep=2pt, itemsep=1pt]
    \item Automatically Authoring Regression Tests for Machine-Learning-Based Systems. \textit{ICSE, 2021}
    \item Progenitor identification and SARS-CoV-2 infection in human distal lung organoids. \textit{Nature, 2020}
    \item Exploratory gene ontology analysis with interactive visualization. \textit{Scientific Reports, 2019}
    \item Visualization and analysis of sc-RNA-seq data by kernel-based similarity learning. \textit{Nature Methods, 2017}
  \end{enumerate}
Full list shown on Google Scholar: \url{https://scholar.google.com/citations?user=2EasRdEAAAAJ&hl}
%-------------------------------------------
\end{document}
