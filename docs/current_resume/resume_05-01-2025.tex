\documentclass[letterpaper,11pt]{article}

\usepackage{latexsym}
\usepackage[empty]{fullpage}
\usepackage{titlesec}
\usepackage{marvosym}
\usepackage[usenames,dvipsnames]{color}
\usepackage{verbatim}
\usepackage{enumitem}
\usepackage[hidelinks]{hyperref}
\usepackage{fancyhdr}
\usepackage[english]{babel}
\usepackage{tabularx}
\usepackage{ragged2e}
\input{glyphtounicode}


%----------FONT OPTIONS-------------------
% sans-serif
% \usepackage[sfdefault]{FiraSans}
% \usepackage[sfdefault]{roboto}
% \usepackage[sfdefault]{noto-sans}
% \usepackage[default]{sourcesanspro}

% serif
% \usepackage{CormorantGaramond}
% \usepackage{charter}


\pagestyle{fancy}
\fancyhf{} % clear all header and footer fields
\fancyfoot{}
\renewcommand{\headrulewidth}{0pt}
\renewcommand{\footrulewidth}{0pt}

% Adjust margins
\addtolength{\oddsidemargin}{-0.5in}
\addtolength{\evensidemargin}{-0.5in}
\addtolength{\textwidth}{1in}
\addtolength{\topmargin}{-.5in}
\addtolength{\textheight}{1.0in}

\urlstyle{same}

\raggedbottom
\raggedright
\setlength{\tabcolsep}{0in}

% Sections formatting
\titleformat{\section}{
  \vspace{-4pt}\scshape\raggedright\large
}{}{0em}{}[\color{black}\titlerule \vspace{-5pt}]

% Ensure that generate pdf is machine readable/ATS parsable
\pdfgentounicode=1

%-------------------------
% Custom commands
\newcommand{\resumeItem}[2]{
  \item\small{
    \textbf{#1}{: #2 \vspace{-2pt}}
  }
}

% Just in case someone needs a heading that does not need to be in a list
\newcommand{\resumeHeading}[4]{
    \begin{tabular*}{0.99\textwidth}[t]{l@{\extracolsep{\fill}}r}
      \textbf{#1} & #2 \\
      \textit{\small#3} & \textit{\small #4} \\
    \end{tabular*}\vspace{-5pt}
}

\newcommand{\resumeSubheading}[4]{
  \vspace{-1pt}\item
    \begin{tabular*}{0.97\textwidth}[t]{l@{\extracolsep{\fill}}r}
      \textbf{#1} & #2 \\
      \textit{\small#3} & \textit{\small #4} \\
    \end{tabular*}\vspace{-5pt}
}

\newcommand{\resumeSubSubheading}[2]{
    \begin{tabular*}{0.97\textwidth}{l@{\extracolsep{\fill}}r}
      \textit{\small#1} & \textit{\small #2} \\
    \end{tabular*}\vspace{-5pt}
}

\newcommand{\resumeSubItem}[2]{\resumeItem{#1}{#2}\vspace{-4pt}}

\renewcommand{\labelitemii}{$\circ$}

\newcommand{\resumeSubHeadingListStart}{\begin{itemize}[leftmargin=*]}
\newcommand{\resumeSubHeadingListEnd}{\end{itemize}}
\newcommand{\resumeItemListStart}{\begin{itemize}}
\newcommand{\resumeItemListEnd}{\end{itemize}\vspace{-5pt}}

%-------------------------------------------
%%%%%%  CV STARTS HERE  %%%%%%%%%%%%%%%%%%%%%%%%%%%%


\begin{document}

%----------HEADING-----------------
\begin{tabular*}{\textwidth}{l@{\extracolsep{\fill}}r}
  \textbf{\href{}{\Large Jason (Junjie) Zhu, Ph.D.}} & Email: \href{}{junjie.zhu.jason@gmail.com}\\
  \href{}{jasonjunjiezhu.com} & Mobile: \href{}{650-285-7123} \\
\end{tabular*}

%-----------SUMMARY-----------------

\section{Summary}
\begin{justify}
I am a curiosity-driven, scientifically trained builder with 10+ years of experience in AI/ML, statistics, and graph algorithms. 
I have had the fortune to collaborate with world-class researchers and top-tier product teams to drive meaningful, collective impact—reflected in 10,000+ citations to my publications. 
Passionate about complex challenges and high-agency environments, I architect and implement scalable solutions across emerging domains, from multi-modal RAGs and intelligent search to biomedical discovery.
\end{justify}

%-----------EDUCATION-----------------
\section{Education}
\begin{tabular*}{\textwidth}{@{\extracolsep{\fill}}p{0.65\textwidth}r}
\textbf{Stanford University} & Stanford, CA \\
\textit{\small Ph.D. in Electrical Engineering · M.S. in Statistics} & \textit{\small 2014 -- 2020} \\[0.4em]
\textbf{Olin College of Engineering} & Needham, MA \\
\textit{\small B.S. in Electrical and Computer Engineering} & \textit{\small 2010 -- 2014} \\
\end{tabular*}
%-----------PROFESSIONAL EXPERIENCE-----------------
\section{Experience}
\begin{tabular*}{0.97\textwidth}[t]{l@{\extracolsep{\fill}}r}
  \textbf{Nexa AI} & Cupertino, CA \\
  \textit{\small Head of AI/ML} & \textit{\small Feb 2025 -- Present} \\
  \multicolumn{2}{l}{
    \begin{minipage}{\textwidth}
      \vspace{0.3em}
      \begin{itemize}[leftmargin=*, itemsep=-4.5pt, topsep=0pt, label={\raisebox{0.4ex}{\tiny\textbullet}}]
        \item  Leadership: Spearheading a small team to optimize Gen-AI edge inference across CPU/NPU/GPU platforms.
        \item  Local RAGs: Pioneering privacy-first RAG systems with lightweight AI models and edge vision processing.
        \item  Agentic Systems: Architecting next-generation AI applications using emerging protocols (e.g., MCP, A2A).
      \end{itemize}
    \end{minipage}
  } \\
  \noalign{\vspace{0.7em}}

  \textbf{Apple} & Cupertino, CA \\
  \textit{\small Machine Learning Engineer} & \textit{\small Jan 2020 -- Feb 2025} \\
  \multicolumn{2}{l}{
    \begin{minipage}{\textwidth}
      \vspace{0.3em}
      \begin{itemize}[leftmargin=*, itemsep=-4.5pt, topsep=0pt, label={\raisebox{0.4ex}{\tiny\textbullet}}]       
        \item  Mentorship: Empowered engineers to achieve recognition through internal and external conference publications.
        \item  Synthetic Data Generation: Pioneered robust frameworks using advanced dimensional perturbation techniques.
        \item  Preference Learning: Designed offline A/B testing to capture preference dynamics and distribution shifts.
        \item  Query Intent Testing: Built high-performance Java pipelines for Maps Search query understanding evaluation.
        \item  Ranking Triage: Proposed and implemented linear-time algorithms to debug and improve multi-ranker systems.
      \end{itemize}
    \end{minipage}
  } \\
  \noalign{\vspace{0.7em}}

  \textbf{Stanford University} & Stanford, CA \\
  \textit{\small Research Assistant} & \textit{\small Sep 2014 -- Feb 2020} \\
  \multicolumn{2}{l}{
    \begin{minipage}{\textwidth}
      \vspace{0.3em}
      \begin{itemize}[leftmargin=*, itemsep=-4.5pt, topsep=0pt, label={\raisebox{0.4ex}{\tiny\textbullet}}]
        \item  Graph Visualization: Engineered interactive visualizations for Gene Ontology analysis.
        \item  Statistical Inference: Enhanced multiple-hypothesis testing to eliminate data snooping bias.
        \item  Unsupervised Learning: Developed novel dimension-reduction techniques for cellular modeling.
        \item  Selective Inference: Implemented advanced methods for tissue-specific eQTL analysis.
        \item  Sequence Alignment: Enhanced DNA sequence alignment performance through C/C++ optimization.
      \end{itemize}
    \end{minipage}
  } \\
  \noalign{\vspace{0.7em}}

  \textbf{Olin College of Engineering} & Needham, MA \\
  \textit{\small Research Assistant} & \textit{\small Sep 2010 -- May 2014} \\
  \multicolumn{2}{l}{
    \begin{minipage}{\textwidth}
      \vspace{0.3em}
      \begin{itemize}[leftmargin=*, itemsep=-4.5pt, topsep=0pt, label={\raisebox{0.4ex}{\tiny\textbullet}}]
        \item  Graph Theory: Resolved distance-2 graph coloring challenges for specialized graph families.
        \item  Information Theory: Developed stochastic geometric models for wireless network interference.
      \end{itemize}
    \end{minipage}
  } \\
\end{tabular*}

%-----------SELECTED PUBLICATIONS-----------------
\section{Selected Publications}
\vspace{0.5em}
  \begin{enumerate}[leftmargin=*, topsep=4pt, parsep=2pt, itemsep=1pt]
    \item Automatically Authoring Regression Tests for Machine-Learning-Based Systems. \textit{ICSE, 2021}
    \item Progenitor identification and SARS-CoV-2 infection in human distal lung organoids. \textit{Nature, 2020}
    \item Exploratory gene ontology analysis with interactive visualization. \textit{Scientific Reports, 2019}
    \item Visualization and analysis of sc-RNA-seq data by kernel-based similarity learning. \textit{Nature Methods, 2017}
  \end{enumerate}
Full list shown on Google Scholar: \url{https://scholar.google.com/citations?user=2EasRdEAAAAJ&hl}
%-------------------------------------------
\end{document}
